\chapter{Control of the interaction}\label{ch:interaction-control}

In this chapter we will analyze how the robot interact with the environment, and we will develop techniques to perform this task based on the control of the forces acting on the robot (generally on the end effector).

This kind of study is essential for the robots will interact with humans (collaborative robots).

\paragraph{Passive control of compliance}

In some tasks to the robots are requested a certain degree of flexibility in contrast to the request of the trajectory requested is strictly followed ($\x(t)=\xd(t)$), for example in an assembly task where it is required to insert an element in a hole (the alignment of end effector with the hole might be not perfect), in this case the problem can be bypassed equipping the robot with an \textbf{Remote Center of Compliance}\index{Remote Center of Compliance}\footnote{\url{https://en.wikipedia.org/wiki/Remote_Center_Compliance}}.
The RCC placed between the robot's wrist and the gripper introduces a form of compliance for the axis perpendicular to the approach one, correcting misalignment in a passive way.

More flexibility behaviour can be gotten with an active control.

\paragraph{Forces measurements}

The measurements of forces and moments are provided by forces sensors that return the measurements of these alone the three axis bound on a local frame (generally in the proximity of the end effector).

The forces sensors are generally based on \textbf{strain gauges}\index{strain gauges}\footnote{\url{https://en.wikipedia.org/wiki/Strain_gauge}} (a device that changes its conductance in function of strain).
The strain gauges are suitable mounted in way to allow the measure of the six component of forces and moments.

\section{Forces in statics}

Let us start from the study of the robots' statics subjected to forces (and moments) acting on the end-effector.
We will use the principle of the virtual works.
For the torques joints we find the contribution as

\[
	dW_\tau = \vect\tau^\trans d\q
\]

and for the forces on the end effector

\[
	dW_\gamma = \vect f^\trans d\vect p_e + \vect\mu^\trans \vect\omega_e dt
\]

with $\vect f$ and $\vect\mu$ are respectively the resulting force and moment act on the end effector.
Exploiting the \textbf{geometrical Jacobian} we can write $dW_\gamma$ as a function of $\q$

\[
	dW_\gamma = \vect f^\trans \J_P(\q) d\q + \vect\mu^\trans \J_O(\q) d\q = \vect\gamma^\trans \J(\q) d\q
\]

where the resulting force and moment are written in the vector $\vect\gamma = \begin{bmatrix} \vect f^\trans & \vect\mu^\trans \end{bmatrix}^\trans$.
The elementary movement and the virtual movement coincide, so we can write

\begin{align*}
	\delta W_\tau &= \vect\tau^\trans \delta\q \\
	\delta W_\gamma &= \vect\gamma^\trans \J(\q) \delta\q
\end{align*}

The robot is in a static state if $\delta W_\tau = \delta W_\gamma$, that is satisfied with

\[
	\vect\tau = \J^\trans(\q) \vect\gamma
\]

\begin{nb}the static relation shows a duality with the robot kinetics that can be written in the form $\vect v_e = \J(\q) \dq$ (\textbf{kineto-static duality}\index{kineto-static duality})\end{nb}

\begin{nb}if $\vect\gamma \in \ker(\J^\trans)$, then it does not require any joint torque to balance $\vect\gamma$\end{nb}


\section{The concept of impedance}

With a generalized approach we can state that for a dynamical system a power \textbf{flow} tent to change a generalized \textbf{effort}.
We can easily see this behaviour in an electrical system, where the flow is the current and the effort is the voltage, the relation between these two measure is expressed by the impedance.
In a mechanical system we can find this duality with \textbf{force} (flow) and the \textbf{position} (effort).
So we can define the mechanical impedance and design a control system based on it.


Let us consider a 1 dof mass ($M$) on which acting two forces ($u, f$) (where $a$ is the mass' acceleration)

\[
	M a = u + f
\]

we introduce a couple spring-dumper acting on the mass through $u$

\[
	u = - k_d v - K_e p
\]

where we indicted with $v, p$ respectively the mass' speed and position.
So the system become

\[
	M a + k_d v + K_e p = f
\]

So we defined a relation between the force and the position (and its derivation) in a \textbf{mass-spring-damper} system;
we will call this relation \text{mechanical impedance}.

\subsection{Dynamics model with external forces}

Let us try to extend the concept of mechanical impedance to the robots control.
We consider the dynamical system with a $\vect\gamma$ force (and moment) acting on the end effector

\[
	\B(\q)\ddq + \C(\q,\dq)\dq + \g(\q) = \vect\tau - \J^\trans(\q)\vect\gamma
\]

and we consider the control law based on the inverse dynamics

\[
	\vect\tau = \B(\q)\y + \C(\q,\dq)\dq + \g(\q)
\]

if we substitute it we get

\[
	\ddq = \y - \B^\inv(\q)\J^\trans(\q)\vect\gamma
\]

now, assume for $\y$ the control law we saw for the inverse dynamic control in operative space (\autoref{eq:control-law-inverse-dynamics-operational-space}), for the closed loop we get

\begin{align*}
	\ddq &= \J_A^\inv(\q) \left( \ddxd + \K_D\dxe + \K_P \xe - \dJ_A(\q,\dq)\dq \right) - \B^\inv(\q)\J^\trans(\q)\vect\gamma \\
	\J_A(\q)\ddq + \dJ_A(\q,\dq)\dq &= \ddxd + \K_D\dxe + \K_P \xe - \J_A(\q)\B^\inv(\q)\J^\trans(\q)\vect\gamma \\
	\intertext{recognizing $\J_A\ddq + \dJ_A(\q,\dq)\dq = \ddx$}
	\ddx &= \ddxd + \K_D\dxe + \K_P \xe - \J_A(\q)\B^\inv(\q)\J^\trans(\q)\vect\gamma \\
	\ddxe + \K_D\dxe + \K_P \xe &= \J_A(\q)\B^\inv(\q)\J^\trans(\q)\vect\gamma \\
	\intertext{defining $\J^\trans(\q)\vect\gamma = \J_A^\trans(\q)\vect\gamma_A$}
	\ddxe + \K_D\dxe + \K_P \xe &= \J_A(\q)\B^\inv(\q)\J_A^\trans(\q)\vect\gamma_A \\
	\intertext{and introducing $\B_A(\q) = \J_A^{-\trans}(\q)\B(\q)\J_A^\inv(\q)$}
	\ddxe + \K_D\dxe + \K_P \xe &= \B_A^\inv(\q)\vect\gamma_A
\end{align*}

we get an impedance relation, which is coupled and only partially assignable.
But if we have also the forces' measurement we can change the control laws to exploit them

\begin{align*}
    \vect\tau &= \B(\q)\y + \C(\q,\dq)\dq + \g(\q) + \J^\trans(\q)\gamma \\
    \y &= \J_A^\inv(\q) \matr M_d^\inv \left( \matr M_d \ddxd + \matr D_d\dxe + \K_d \xe - \matr M_d \dJ_A(\q,\dq)\dq - \vect\gamma_A \right)
\end{align*}

with $\matr M_d, \matr D_d, \K_d$ diagonal positive definite matrices;
the closed loop become

\[
	\matr M_d \ddxe + \matr D_d \dxe + \K_d \xe = \vect\gamma_A
\]

so, a completely decoupled system.

This form defines a \textbf{mechanical impedance} between the forces ( and moments) and the position error in the operational space.

\begin{nb}the \textbf{mechanical impedance} have the same shape of a mass-spring-damper system  with $\matr M_d$ mass, $\matr D_d$ damping and $\K_d$ stiffness\end{nb}

Unluckily this method impose several constraints:

\begin{itemize}
	\item it requires a complete knowledge of the dynamic model, it does not guarantee a compensation of the model errors
	\item it requires complete access to the robot, the control law is design directly on joints torques
	\item the system become inherently compliant to external disturbances, which is conflicting with the typical stiffness required to the industrial robots
\end{itemize}

\subsection{Admittance control}

Defining the \textbf{admittance filter} function (the inverse of the impedance)

\[
	H(s) = \frac{1}{\matr M_d s^2 + \matr D_d s + \K_d}
\]

we can make the control system in \autoref{fig:admittance-control-interaction}.
In which

\[
	\xd - \x \approx \frac{1}{\matr M_d s^2 + \matr D_d s + \K_d} \vect f_{ext}
\]

\begin{figure}[htb]
	\centering
	\resizebox{0.7\textwidth}{!}{
		\begin{tikzpicture}
			\node [input] (ix_d) {};

			\node [sum, right=1cm of ix_d] (sum_f) {};
			\node [sum, right=1cm of sum_f] (sum_e) {};

			\node [block, right=1cm of sum_e] (controller) {High gain\\motion\\controller};
			\node [block, right=1cm of controller] (robot) {Robot};
			\node [block, above left=0.8cm and -0.5cm of controller] (admittance) {$H(s)$};

			\node [output, right=1cm of robot] (ox) {};

			\draw [->] (ix_d) -- node [pos=0.3] {$\xd$} (sum_f);
			\draw [->] (sum_f) -- (sum_e);
			\draw [->] (sum_e) -- (controller);
			\draw [->] (controller) -- (robot);
			\draw [->] (robot) -- node [pos=0.8] {$\x$} node [node, name=x, pos=0.5] {} (ox);

			\draw [->] (robot) |- node[pos=0.2, right] {$\vect f_{ext}$} (admittance);
			\draw [->] (admittance) -| (sum_f);

			\draw [->] (x) -- ++(0,-1.5) -| node [pos=0.95] {$-$} (sum_e);
		\end{tikzpicture}
	}
	\caption{Admittance control}
	\label{fig:admittance-control-interaction}
\end{figure}

\section{Natural constraints and artificial constraints}

It is generally impossible to assign simultaneously an arbitrary position and force to a system, but in some task a control both on position and force is required, to handle this requirement we define a new concept based on constraints.

In the 3d space (with reference to a frame) we can identify 6 dof for the generalized movement velocity, 3 for the lineal speed ($v_x,v_y,v_z$) and 3 for the rotational speed ($\omega_x,\omega_y,\omega_z$) to which we can associate 6 generalized forces (or moments), 3 linear forces ($f_x,f_y,f_z$) and 3 torques ($\mu_x,\mu_y,\mu_z$).

So, we can define two kinds of constraints on this 12 variables:

We will call \textbf{natural constraints} ones that impose a limitation on a generalized movement velocity and forces, these are the constraints imposed by the environment.
\\ \begin{em}e.g. a rigid plane impose a limit on the linear speed $v_n$ in the direction defines by the plane normal, instead the absence of an obstacle in the generic direction of vector unit $\vect u$ (that is opposed to the motion) can not allow the imposition of the associated linear force $f_u$.\end{em}

We will call \textbf{artificial constraints} ones we can impose to the system arbitrarily
\\ \begin{em}e.g. a pipe in a hole allows to impose an artificial constraints on the rotation velocity $\omega_l$ of the pipe in the hole (where $\vect l$ is vector unit in the direction of the pipe longitudinal axis)\end{em}

It is evident that the artificial and natural constraints are complementary, and they suggest the control approach where the desire functions are defined only for the \textbf{artificial constraints}, while the state subject to the \textbf{natural constraints} are free to evolve.

So we can define the useful diagonal matrix $\matr \Sigma$ ($6\times6$) with only zeros or ones as elements called \textbf{selection matrix}\index{selection matrix}.

The $\matr \Sigma$ matrix is defined to weight the error on the reference for controlled forces the element (i,i) of the matrix is $1$ for the others $0$.

An approximate scheme of the control based on this approach can see in \autoref{fig:hybrid-control-interaction}, it is called \textbf{hybrid force/position control} and it is characterized by a high flexibility.
It is possible identify some inconsistencies can appear due to nominal model used:

\begin{itemize}
	\item a detected force in nominally free direction caused by a friction at the contact
	\item a detected movement speed in nominally constrained direction caused by compliance in the robot structure or contact
	\item uncertainty in the environment geometry at the contact
\end{itemize}

The first two are automatically filtered by the select matrix, the third can be mitigated by real time estimation process.

\begin{figure}[htb]
	\centering
	%\resizebox{0.7\textwidth}{!}{
		\begin{tikzpicture}
			\node [input] at (0,1) (ifd) {};
			\node [input] at (0,-1) (idxd) {};

			\node [block, right=1 of ifd] (sel_f) {$\matr \Sigma$};
			\node [block, right=1 of idxd] (sel_dx) {$\I - \matr \Sigma$};
			\draw [->] (ifd) -- node[pos=0.1] {$\reference{\vect\gamma}$} (sel_f);
			\draw [->] (idxd) -- node[pos=0.1] {$\dxd$} (sel_dx);

			\node [block, right=1 of sel_f] (control_f) {Force\\control};
			\node [block, right=1 of sel_dx] (control_dx) {Motion\\control};
			\draw [->] (sel_f) -- (control_f);
			\draw [->] (sel_dx) -- (control_dx);

			\node [block, above=0.5 of control_f] (sel_ff) {$\matr \Sigma$};
			\node [block, below=0.5 of control_dx] (sel_dxf) {$\I - \matr \Sigma$};
			\draw [->] (sel_ff) -- (control_f);
			\draw [->] (sel_dxf) -- (control_dx);

			\node [sum] at ($(control_f)!0.5!(control_dx) + (2,0)$) (sum) {};
			\draw [->] (control_f) -| (sum);
			\draw [->] (control_dx) -| (sum);

			\node [output, right=1 of sum] (u) {};
			\draw [->] (sum) -- node[pos=0.9] {$\vect\tau$} (u);

			\node [input] at (u |- sel_ff) (iff) {};
			\node [input] at (u |- sel_dxf) (idxf) {};
			\draw [->] (iff) -- node[pos=0.1] {$\estimate{\vect\gamma}$} (sel_ff);
			\draw [->] (idxf) -- node[pos=0.1] {$\dxs$} (sel_dxf);
		\end{tikzpicture}
	%}
	\caption{Hybrid force/position control}
	\label{fig:hybrid-control-interaction}
\end{figure}

\section{Explicit control}

Consider a 1dof system with a mass $m$ subject to 2 forces $f, u$ subjected to the relation (where $x$ is the position of the mass)

\[
	  w + u = m \ddot{x}
\]

if we introduce a spring with law $-k x$ on $w$ we get

\[
	-k x + u = m \ddot{x}
\]

in the \textbf{Laplace} form

\[
x(s) = \frac{1}{m s^2 + k} u(s)
\]

so if we evaluated the reaction force on the other side $f$ of the spring we can get the relation

\[
	f(s) = \frac{k}{m s^2 + k} u(s)
\]

an equivalent block diagram can be seen in the \autoref{fig:spring-block-control-interaction}

\begin{figure}[htb]
	\centering
	%\resizebox{0.7\textwidth}{!}{
	\begin{tikzpicture}[node distance=1cm]
		\node [input] (iu) {};

		\node [sum, right=of iu] (sum) {};
		\draw [->] (iu) -- node[pos=0.2] {$u$} (sum);

		\node [block, right=0.6 of sum] (g) {$\frac{1}{m s^2}$};
		\draw [->] (sum) -- (g);

		\node [block, right=of g] (k) {$k$};
		\draw [->] (g) -- node {$x$} (k);

		\node [output, right=of k] (of) {};
		\draw [->] (k) -- node[node] (f) {} node[pos=0.8] {$f$} (of);
		\draw [->] (f) -- ++(0,1) -| node[pos=0.8] {$-$} (sum);
	\end{tikzpicture}
	%}
	\caption{Spring-mass-environment interaction equivalent block diagram}
	\label{fig:spring-block-control-interaction}
\end{figure}

The transfers function $\frac{F(s)}{U(s)}$ can be also written as

\[
	G_f(s) = \frac{\frac{k}{m}}{s^2 + \frac{k}{m}} = \frac{\omega_n^2}{s^2 + \omega_n^2}
\]

highlighting it presents 2 imaginary poles in $\pm i \sqrt{k/m}$ and for $k \to \infty$ then $G_f(s) \to 1$

So we can close a loop around the force to regulate it (\autoref{fig:explicit-control-interaction})

\begin{figure}[htb]
	\centering
	%\resizebox{0.7\textwidth}{!}{
	\begin{tikzpicture}
		\node [input] (if) {};

		\node [sum, right=of if] (sum_e) {};
		\draw [->] (if) -- node[pos=0.2] {$\reference{f}$} (sum_e);

		\node [block, right=0.6 of sum_e] (r) {$R_f(s)$};
		\draw [->] (sum_e) -- (r);

		\node [sum, right=of r] (sum) {};
		\draw [->] (r) -- node {$u$} (sum);

		\node [block, right=0.6 of sum] (g) {$\frac{1}{m s^2}$};
		\draw [->] (sum) -- (g);

		\node [block, right=of g] (k) {$k$};
		\draw [->] (g) -- node {$x$} (k);

		\node [output, right=of k] (of) {};
		\draw [->] (k) -- node[node] (f) {} node[pos=0.8] {$f$} (of);
		\draw [->] (f) -- ++(0,1) -| node[pos=0.8] {$-$} (sum);
		\draw [->] (f) -- ++(0,-1) -| node[pos=0.9] {$-$} (sum_e);
	\end{tikzpicture}
	%}
	\caption{Explicit force control block diagram}
	\label{fig:explicit-control-interaction}
\end{figure}

a good choice for $R_f(s)$ is a pure integrator

\[
	R_f(s) = \frac{K_{if}}{s}
\]

so we can directly assign $\omega_{cf}$ changing $K_{if} \approx \omega_{cf}$ if $\omega_{cf} < \omega_n$; with this choice the resonant frequency $\omega_n$ is out of the system bandwidth.

\section{Implicit control}

Now let us start to analyze the same system saw for the explicit control when a control loop on position is already closed (\autoref{fig:closed-position-control-interaction}).

\begin{figure}[htb]
	\centering
	%\resizebox{0.7\textwidth}{!}{
	\begin{tikzpicture}
		\node [input] (ix) {};

		\node [sum, right=of ix] (sum_e) {};
		\draw [->] (ix) -- node[pos=0.2] {$\reference{x}$} (sum_e);

		\node [block, right=0.6 of sum_e] (r) {$R(s)$};
		\draw [->] (sum_e) -- (r);

		\node [sum, right=of r] (sum) {};
		\draw [->] (r) -- node {$u$} (sum);

		\node [block, right=0.6 of sum] (g) {$G(s)$};
		\draw [->] (sum) -- (g);

		\node [block, right=of g] (k) {$k$};
		\draw [->] (g) -- node[node] (x) {} node {$x$} (k);
		\draw [->] (x) -- ++(0,-1) -| node[pos=0.9] {$-$} (sum_e);

		\node [output, right=of k] (of) {};
		\draw [->] (k) -- node[node] (f) {} node[pos=0.8] {$f$} (of);
		\draw [->] (f) -- ++(0,1) -| node[pos=0.8] {$-$} (sum);
	\end{tikzpicture}
	%}
	\caption{Closed loop on position}
	\label{fig:closed-position-control-interaction}
\end{figure}

The corresponding closed loop is

\[
	\frac{f(s)}{\reference{x}(s)} = \frac{kR(s)G(s)}{1 + kG(s) + R(s)G(s)}
\]

which for infinitely contact stiffness ($k \to \infty$) become $\frac{f(s)}{\reference{x}(s)}=R(s)$ and $f \approx u$

\begin{figure}[htb]
	\centering
	%\resizebox{0.7\textwidth}{!}{
	\begin{tikzpicture}
		\node [input] (if) {};
		\node [block, right=of ix] (r) {$R(s)$};
		\draw [->] (ix) -- node[pos=0.2] {$\reference{x}$} (r);

		\node [block, right=of r] (g) {$\approx 1$};
		\draw [->] (r) -- node {$u$} (g);

		\node [output, right=of g] (of) {};
		\draw [->] (g) -- node[pos=0.8] {$f$} (of);
	\end{tikzpicture}
	%}
	\caption{Closed loop on position with infinite contact stiffness}
	\label{fig:stiffness-closed-position-control-interaction}
\end{figure}

now, let us try to close a loop on the force with another controller.
In particular, we can design a regulator composed by two action, one to compensate the position regulator influence ($C(s)$), and another one to impose the desired dynamics to $f$ ($R_f(s)$), you can see this system in \autoref{fig:explicit-block-control-interaction}.

\begin{figure}[htb]
	\centering
	%\resizebox{0.7\textwidth}{!}{
	\begin{tikzpicture}
		\node [input] (ix) {};

		\node [sum, right=of ix] (sum) {};
		\draw [->] (ix) -- node[pos=0.2] {$\reference{f}$} (sum);

		\node [block, right=0.6 of sum] (rf) {$R_f(s)$};
		\draw [->] (sum) -- (rf);

		\node [block, right=of rf] (c) {$C(s)$};
		\draw [->] (rf) -- (c);

		\node [block, right=of c] (r) {$R(s)$};
		\draw [->] (c) -- node {$\reference{x}$} (r);

		\node [block, right=of r] (g) {$\approx 1$};
		\draw [->] (r) -- node {$u$} (g);

		\node [output, right=of g] (of) {};
		\draw [->] (g) -- node[node] (f) {} node[pos=0.8] {$f$} (of);
		\draw [->] (f) -- ++(0,-1) -| node[pos=0.8] {$-$} (sum);
	\end{tikzpicture}
	%}
	\caption{Closed loop on position with infinite contact stiffness}
	\label{fig:explicit-block-control-interaction}
\end{figure}

Under the hypothesis that the position regulator is a PID

\[
	R(s) = k_P + k_I\frac{1}{s} + k_D s = \frac{k_D s^2 + k_P s + k_I}{s}
\]

we can design the compensator as

\[
	C(s) = \frac{1}{k_D s^2 + k_P s + k_I}
\]

so the open loop function become (\autoref{fig:compesanted-implicit-block-control-interaction})

\[
	L_f(s) = R_f(s) C(s) R(s) = \frac{1}{s} R_f(s)
\]

\begin{figure}[htb]
	\centering
	%\resizebox{0.7\textwidth}{!}{
	\begin{tikzpicture}
		\node [input] (if) {};

		\node [sum, right=of if] (sum) {};
		\draw [->] (if) -- node[pos=0.2] {$\reference{f}$} (sum);

		\node [block, right=0.6 of sum] (rf) {$R_f(s)$};
		\draw [->] (sum) -- (rf);

		\node [integrator, right=of rf] (int) {};
		\draw [->] (rf) -- (int);

		\node [output, right=of int] (of) {};
		\draw [->] (int) -- node[node] (f) {} node[pos=0.8] {$f$} (of);
		\draw [->] (f) -- ++(0,-1) -| node[pos=0.8] {$-$} (sum);
	\end{tikzpicture}
	%}
	\caption{Compensated implicit force control}
	\label{fig:compesanted-implicit-block-control-interaction}
\end{figure}

We can take $R_f(s)$ as PI controller

\[
	R_f(s) = k_{P_f} + k_{I_f} \frac{1}{s}
\]

so finally the open loop function become

\[
	L_f(s) = \frac{s k_{P_f} + k_{I_f}}{s^2}
\]

the $\omega_{c_f}$ can be set as $\omega_{c_f} \approx k_{P_f}$
