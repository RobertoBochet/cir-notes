\chapter{Decentralized control}

In the decentralized control each joint is controlled independently from each others

\section{Simplified dynamic model}

$$ \matr{B}(\q)\ddq + \matr{C}(\q,\dq)\dq + \vect{g}(\q) = \vect{\tau} $$

In the model each joint have a motor; to simplify we consider just the effect related to its spinning around its own axis and a rigid transmission.

$$ J_{mi}\ddot{q}_{mi} + D_{mi}\dot{q}_{mi} = \tau_{mi} - \tau_{lmi} \quad i=1,\dotso,n $$
$$ \tau_{lmi} = \tau_{li} / n_i $$

\begin{conditions}
J_{mi} & moment of inertial of the motor \\
D_{mi} & friction viscous coefficient of the motor \\
\tau_{lmi} & load torque on the axis of the motor \\
\tau_{li} & load torque on the axis of the joint \\
n_i & transmission reduction rate between motor and joint
\end{conditions}

\warning{missing dynamics model derivation}

\section{Electrical motor}

Let's consider a simple DC motor

\begin{align*}
    V(t) &= R i(t) + L \dot{i}(t) + E(t) \\
    E(t) &= K \dot{q}_m(t) \\
    \tau_m(t) &= K i(t) = J_m \ddot{q}_m(t)
\end{align*}

\begin{figure}
\centering
\resizebox{\textwidth}{!}{
\begin{tikzpicture}[auto,>=latex', node distance=1cm]
    \node [input] (input) {};
    \node [sum, right= 1cm of input] (sum_v) {};
    \node [block, right=0.6cm of sum_v] (v2i) {$\frac{1}{sL+R}$};
    \node [block, right=1cm of v2i] (i2tau) {$K$};
    \node [block, above=0.5cm of i2tau] (dq2e) {$K$};
    \node [block, right=1cm of i2tau] (tau2dq) {$\frac{1}{J_m s}$};
    \node [integrator, right=1cm of tau2dq] (dq2q) {};
    \node [output, right=1cm of dq2q] (output) {};
    
    \draw [draw,->] (input) -- node {$V$} (sum_v);
    \draw [->] (sum_v) -- (v2i);
    \draw [->] (v2i) -- node {$i$} (i2tau);
    \draw [->] (i2tau) -- node {$\tau_m$} (tau2dq);
    \draw [->] (tau2dq) -- node [pos=0.35] {$\dot{q}_m$} node [name=dq, pos=0.65,inner sep=0] {}  (dq2q);
    \draw [->] (dq2q) -- node {$q_m$} (output);
    \draw [->] (dq.0) |- (dq2e);
    \draw [->] (dq2e) -| node [pos=0.95] {$-$} node [near end] {$E$} (sum_v);
\end{tikzpicture}
}
\caption{DC motor model}
\label{fig:motor_model}
\end{figure}

\subsubsection{Control of the current}

We can close an high frequency (thousands $\rad/s$) loop for the current as it can been seen in \autoref{fig:motor_current_control}. $E$ is considered to change as a slowly varying disturbance, that the controller can effectively reject.
From the point of view of the mechanical system the current loop is so fast to consider the current change is instantaneous. So for the mechanical control we can consider to control directly the motor torque

$$\tau_m(t) = Ki(t) \approx K \bar{i}(t)$$

\begin{figure}
\centering
\resizebox{\textwidth}{!}{
\begin{tikzpicture}[auto,>=latex', node distance=1cm]
    \node [input] (input) {};
    \node [sum, right= 1cm of input] (sum_e) {};
    \node [block, right=0.6cm of sum_e] (e2v) {$R_i(s)$};
    \node [sum, right= 0.6cm of e2v] (sum_v) {};
    \node [block, right=0.6cm of sum_v] (v2i) {$\frac{1}{sL+R}$};
    \node [block, right=1cm of v2i] (i2tau) {$K$};
    \node [block, above=0.5cm of i2tau] (dq2e) {$K$};
    \node [block, right=1cm of i2tau] (tau2dq) {$\frac{1}{J_m s}$};
    \node [integrator, right=1cm of tau2dq] (dq2q) {};
    \node [output, right=1cm of dq2q] (output) {};
    
    \draw [draw,->] (input) -- node {$\bar{i}$} (sum_e);
    \draw [->] (sum_e) -- (e2v);
    \draw [->] (e2v) -- (sum_v);
    \draw [->] (sum_v) -- (v2i);
    \draw [->] (v2i) -- node [name=i] {$i$} (i2tau);
    \draw [->] (i) -- ++ (0,-1.5) -| node [pos=0.95] {$-$} (sum_e);
    \draw [->] (i2tau) -- node {$\tau_m$} (tau2dq);
    \draw [->] (tau2dq) -- node [pos=0.35] {$\dot{q}_m$} node [name=dq, pos=0.65,inner sep=0] {}  (dq2q);
    \draw [->] (dq2q) -- node {$q_m$} (output);
    \draw [->] (dq.0) |- (dq2e);
    \draw [->] (dq2e) -| node [pos=0.95] {$-$} node [near end] {$E$} (sum_v);
\end{tikzpicture}
}
\caption{motor current control}
\label{fig:motor_current_control}
\end{figure}

\section{Rigid transmission approximation}

For a rigid transmission we have the following system

$$ J_m\ddot{q}_m + d_m \dot{q}_m = \tau_m - \tau_{lm} $$
$$ J_l\ddot{q}_l = n \tau_{lm} - \tau_l $$
$$ q_m = n q_l $$

The first and the second equation can be merged exploiting the third equation

$$ \left(J_m + \frac{J_l}{n^2}\right) \ddot{q}_m + d_m \dot{q}_m = \tau_m - \frac{\tau_l}{n} $$

The result model can been seen in the \autoref{fig:mechanical_system_rigid}

$$ G_v(s) = \frac{1}{d_m + \left(J_m + \frac{J_l}{n^2}\right) s} = \frac{1/d_m}{1 + \frac{J_m + J_l/n^2}{d_m} s}  $$

$d_m$ is an uncertain small parameter, since $d_m$ give a real stable pole we can consider the most conservative situation in which $d_m=0$, so we get

$$ G_v(s) = \frac{1}{\left(J_m + \frac{J_l}{n^2}\right) s} $$

\begin{figure}
\centering
\resizebox{0.6\textwidth}{!}{%
\begin{tikzpicture}[auto,>=latex']
    \node [input, name=tau_m] {};
    \node [sum, right= 0.5cm of tau_m] (sum_tau) {};
    \node [block, above=0.5cm of sum_tau, scale=0.5] (taul2taulr) {$1/n$};
    \node [input, above=0.5cm of taul2taulr, name=tau_l] {};
    \node [block, right=1cm of sum_tau] (tau2dqm) {$G_v(s)$};
    \node [integrator, right=1cm of tau2dqm] (dqm2qm) {};
    \node [output, right=1cm of dqm2qm] (q_m) {};
    
    \draw [->] (tau_m) -- node {$\tau_m$} (sum_tau);
    \draw [->] (tau_l) -- node {$\tau_l$} (taul2taulr);
    \draw [->] (taul2taulr) -- node [pos=0.95] {$-$} (sum_tau);
    \draw [->] (sum_tau) -- (tau2dqm);
    \draw [->] (tau2dqm) -- node {$\dot{q}_m$} (dqm2qm);
    \draw [->] (dqm2qm) -- node {$q_m$} (q_m);
\end{tikzpicture}
}%
\caption{Mechanical system with rigid transmission}
\label{fig:mechanical_system_rigid}
\end{figure}

\subsection{P/PI control}

In order to control position of the motor we can close in cascade two loop (in addition to the current loop). The first with a \textbf{PI} regulator on the motor speed and the second with an \textbf{P} regulator on the motor position as you can see in \autoref{fig:rigid_control_position}.

\begin{em}
n.b. in \autoref{fig:rigid_control_position} the current loop is omitted because, as we said before, the current loop has an much higher bandwidth than the speed loop so we can assume $\bar{\tau}_m=\tau_m$.
\end{em}

\begin{figure}
\centering
\resizebox{\textwidth}{!}{
\begin{tikzpicture}
    \node [input] (input) {};
    \node [sum, right=1cm of input] (sum_p) {};
    \node [block, right=.5cm of sum_p] (p) {$R_{P}(s)$};
    \node [sum, right=.5cm of p] (sum_v) {};
    \node [block, right=.5cm of sum_v] (pi) {$R_{PI}(s)$};
    \node [sum, right=1cm of pi] (sum_tau) {};
    \node [block, above=.7cm of sum_tau, scale=.7] (taul2taulr) {$1/n$};
    \node [input, above=.5cm of taul2taulr, name=tau_l] {};
    \node [block, right=.5cm of sum_tau] (tau2dqm) {$G_{v}(s)$};
    \node [integrator, right=1cm of tau2dqm] (dqm2qm) {};
    \node [output, right=1cm of dqm2qm] (output) {};
    
    \draw [draw,->] (input) -- node {$\bar{q}_m$} (sum_p);
    \draw [->] (sum_p) -- (p);
    \draw [->] (p) -- (sum_v);
    \draw [->] (sum_v) -- (pi);
    \draw [->] (pi) -- node {$\tau_m$} (sum_tau);
    \draw [->] (tau_l) -- node {$\tau_l$} (taul2taulr);
    \draw [->] (taul2taulr) -- node [pos=0.95] {$-$} (sum_tau);
    \draw [->] (sum_tau) -- (tau2dqm);
    \draw [->] (tau2dqm) -- node [name=dq_m] {$\dot{q}_m$} (dqm2qm);
    \draw [->] (dqm2qm) -- node [name=q_m] {$q_m$} (output);
    \draw [->] (dq_m) -- ++ (0,-1.2) -| node [pos=0.9] {$-$} (sum_v);
    \draw [->] (q_m) -- ++ (0,-1.5) -| node [pos=0.9] {$-$} (sum_p);
\end{tikzpicture}
}
\caption{Control scheme P/PI with a rigid transmission}
\label{fig:rigid_control_position}
\end{figure}

\subsubsection{PI speed control design}

As we stated the P/PI design is done with cascade approach, so as first step we can design the speed loop.

$$
R_{PI}(s) = k_{pv} \frac{1 + t_{iv} s}{s} \implies
L_v(s) = \frac{k_{pv}}{\left(J_m + \frac{J_l}{n^2}\right)} \frac{1 + t_{iv} s}{s^2}
$$

We impose an $\omega_{cv}$ with $k_{pv}$

$$ \omega_{cv}=\frac{k_{pv}}{\left(J_m + \frac{J_l}{n^2}\right)} $$

and to guarantee that the $\omega_{cv}$ desired coincide with the effective one we have to put the zero in low frequency respect it

$$ t_{iv} \approx \frac{1}{(0.1 \div 0.3)\omega_{cv}} $$

With this regulator we get a close loop function for the speed loop 

$$ F_v(s) \approx \frac{1}{1 + \frac{1}{\omega_{cv}}s} $$

\subsubsection{P position control design}

We can now design the $R_P(s)$ controller taking in account the speed closed loop designed above.

$$
R_{P}(s) = k_{pp} \implies
L_p(s) = k_{pp} \frac{1}{s \left(1 + \frac{1}{\omega_{cv}} s\right)}
$$

if we choose $\omega_{cp} \ll \omega_{cv}$ is sufficient to impose $k_{pp} = \omega_{cp}$

The overall closed loop on the position guarantee null static error and a bandwidth of $\omega_{cp}$.

\subsubsection{Speed feed-forward}

A possible improvement of the \textbf{P/PI} scheme is the speed feed-forward scheme \autoref{fig:rigid_control_ff}. The block $k_{ff}s$ improves the speed of the response of the position control. The $k_{ff}$ coefficient is chosen between 0 and 1.

\begin{figure}
\centering
\resizebox{\textwidth}{!}{
\begin{tikzpicture}
    \node [input] (input) {};
    \node [sum, right=1cm of input] (sum_p) {};
    \node [block, right=.5cm of sum_p] (r_p) {$R_{P}(s)$};
    \node [block, above=.5cm of r_p] (ff) {$k_{ff}s$};
    \node [sum, right=.5cm of r_p] (sum_v) {};
    \node [block, right=.5cm of sum_v] (r_v) {$R_{PI}(s)$};
    \node [sum, right=1cm of r_v] (sum_tau) {};
    \node [block, above=.7cm of sum_tau, scale=.7] (taul2taulr) {$1/n$};
    \node [input, above=.5cm of taul2taulr, name=tau_l] {};
    \node [block, right=.5cm of sum_tau] (tau2dqm) {$G_{v}(s)$};
    \node [integrator, right=1cm of tau2dqm] (dqm2qm) {};
    \node [output, right=1cm of dqm2qm] (output) {};
    
    \draw [draw,->] (input) -- node [pos=0.3] {$\bar{q}_m$} node [name=rqm, pos=0.7,inner sep=0] {} (sum_p);
    \draw [->] (sum_p) -- (r_p);
    \draw [->] (r_p) -- (sum_v);
    \draw [->] (rqm) |- (ff);
    \draw [->] (ff) -| (sum_v);
    \draw [->] (sum_v) -- (r_v);
    \draw [->] (r_v) -- node {$\tau_m$} (sum_tau);
    \draw [->] (tau_l) -- node {$\tau_l$} (taul2taulr);
    \draw [->] (taul2taulr) -- node [pos=0.95] {$-$} (sum_tau);
    \draw [->] (sum_tau) -- (tau2dqm);
    \draw [->] (tau2dqm) -- node [name=dq_m] {$\dot{q}_m$} (dqm2qm);
    \draw [->] (dqm2qm) -- node [name=q_m] {$q_m$} (output);
    \draw [->] (dq_m) -- ++ (0,-1.2) -| node [pos=0.9] {$-$} (sum_v);
    \draw [->] (q_m) -- ++ (0,-1.5) -| node [pos=0.9] {$-$} (sum_p);
\end{tikzpicture}
}
\caption{Control scheme P/PI with a rigid transmission and speed feed-forward}
\label{fig:rigid_control_ff}
\end{figure}

\subsubsection{Position measure only}

If only available the position measure but not the speed, it can be derived with a process of differentiation on the position. The control scheme became as \autoref{fig:rigid_control_position_only} that are equivalent to \autoref{fig:rigid_control_position_only_pid} where all the controller block are substituted with a single \textbf{PID} on position loop.


\begin{figure}
\centering
\resizebox{\textwidth}{!}{
\begin{tikzpicture}
    \node [input] (input) {};
    \node [sum, right=1cm of input] (sum_p) {};
    \node [block, right=.5cm of sum_p] (r_p) {$R_{P}(s)$};
    \node [block, above=.5cm of r_p] (ff) {$s$};
    \node [sum, right=.5cm of r_p] (sum_v) {};
    \node [block, below=.5cm of sum_v] (d) {$s$};
    \node [node, below=.5cm of d] (fq_m) {};
    \node [block, right=.5cm of sum_v] (r_v) {$R_{PI}(s)$};
    \node [sum, right=1cm of r_v] (sum_tau) {};
    \node [block, above=.7cm of sum_tau, scale=.7] (taul2taulr) {$1/n$};
    \node [input, above=.5cm of taul2taulr, name=tau_l] {};
    \node [block, right=.5cm of sum_tau] (tau2dqm) {$G_{v}(s)$};
    \node [integrator, right=1cm of tau2dqm] (dqm2qm) {};
    \node [output, right=1cm of dqm2qm] (output) {};
    
    \draw [draw,->] (input) -- node [pos=0.3] {$\bar{q}_m$} node [node, name=rqm, pos=0.7] {} (sum_p);
    \draw [->] (sum_p) -- (r_p);
    \draw [->] (r_p) -- (sum_v);
    \draw [->] (rqm) |- (ff);
    \draw [->] (ff) -| (sum_v);
    \draw [->] (sum_v) -- (r_v);
    \draw [->] (r_v) -- node {$\tau_m$} (sum_tau);
    \draw [->] (tau_l) -- node {$\tau_l$} (taul2taulr);
    \draw [->] (taul2taulr) -- node [pos=0.95] {$-$} (sum_tau);
    \draw [->] (sum_tau) -- (tau2dqm);
    \draw [->] (tau2dqm) -- node [name=dq_m] {$\dot{q}_m$} (dqm2qm);
    \draw [->] (dqm2qm) -- node [name=q_m] {$q_m$} (output);
    \draw [->] (d) -- node [pos=0.5] {$-$} (sum_v);
    \draw [->] (fq_m) -- (d);
    \draw [->] (q_m) |- (fq_m) -| node [pos=0.9] {$-$} (sum_p);
\end{tikzpicture}
}
\caption{Control scheme with only position measurement}
\label{fig:rigid_control_position_only}
\end{figure}

\begin{figure}
\centering
\resizebox{\textwidth}{!}{
\begin{tikzpicture}
    \node [input] (input) {};
    \node [sum, right=1cm of input] (sum_p) {};
    \node [block, right=.5cm of sum_p] (r) {$R_{PID}(s)$};
    \node [sum, right=1cm of r] (sum_tau) {};
    \node [block, above=.7cm of sum_tau, scale=.7] (taul2taulr) {$1/n$};
    \node [input, above=.5cm of taul2taulr, name=tau_l] {};
    \node [block, right=.5cm of sum_tau] (tau2dqm) {$G_{v}(s)$};
    \node [integrator, right=1cm of tau2dqm] (dqm2qm) {};
    \node [output, right=1cm of dqm2qm] (output) {};
    
    \draw [draw,->] (input) -- node [pos=0.3] {$\bar{q}_m$} node [node, name=rqm, pos=0.7] {} (sum_p);
    \draw [->] (sum_p) -- (r);
    \draw [->] (r) -- node {$\tau_m$} (sum_tau);
    \draw [->] (tau_l) -- node {$\tau_l$} (taul2taulr);
    \draw [->] (taul2taulr) -- node [pos=0.95] {$-$} (sum_tau);
    \draw [->] (sum_tau) -- (tau2dqm);
    \draw [->] (tau2dqm) -- node [name=dq_m] {$\dot{q}_m$} (dqm2qm);
    \draw [->] (dqm2qm) -- node [name=q_m] {$q_m$} (output);
    \draw [->] (q_m) -- ++ (0, -1.5) -| node [pos=0.9] {$-$} (sum_p);
\end{tikzpicture}
}
\caption{Control scheme PID with only position measurement}
\label{fig:rigid_control_position_only_pid}
\end{figure}

\section{Elastic transmission}

If we look only to the rigid transmission model we do not find significant limitations in the bandwidth of speed control, but from experimental data we can clearly find several limitation mainly bound to vibration, we have to improve the rigid model to take in account these limitations.

If we consider a elastic transmission we can write the following system

$$ J_m\ddot{q}_m + d_m \dot{q}_m = \tau_m - \tau_{lm} $$
$$ J_l\ddot{q}_l = n \tau_{lm} - \tau_l $$
$$ \tau_{lm} = k_{el}(q_m - n q_l) + d_{el}(\dot{q}_m - n \dot{q}_l) $$

The $G_v(s)$ now is composed from one real pole, two complex poles and two complex zeros; in particular we can find the parameter $\omega_z$ of the complex zeros as

$$\omega_z = \sqrt{\frac{k_{el}}{J_l}}$$

and a reasonable choice for the speed closed loop bandwidth is $\omega_{cv} \approx 0.7 \omega_z$
